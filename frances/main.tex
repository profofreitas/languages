\documentclass[a4paper,10pt]{article}
\usepackage[utf8]{inputenc}

\setlength{\parindent}{0em}
\setlength{\parskip}{1em}
%opening
% \title{}
% \author{}

\begin{document}

% \maketitle

\section{Welcome and bienvenue!}

Salut! Bienvenue!

\section{French Sounds}


a: ch\textbf{a}t, s\textbf{a}lut, (as in 'f\textbf{a}ther')


je: j\textbf{e} (as in 'p\textbf{a}tition')


i: nu\textbf{i}t, p\textbf{i}zza (as in 'L\textbf{i}sa')


o: h\textbf{o}mme, gr\textbf{o}s (as in '\textbf{o}r')


u: t\textbf{u}, sal\textbf{u}t (say 'ee' while puckering your lips)

\section{Gender}
Masculine nouns use \textbf{un}, and feminine nouns use \textbf{une}.


un garçon


une fille


un homme


une femme


un chien


une pizza

\section{I think, therefore}


je \textbf{suis}


tu \textbf{es}


il/elle \textbf{est}


Examples:


Je suis un garçon

\section{Hello there!}


\textbf{Bonjour!}: Good morning!/ Good afternoon!


\textbf{Bonsoir!}: Good evening!


\textbf{Salut}: Hi!/Good bye! (If you're being casual, you can say \textit{Salut !}
 any time of day!)

 \section{How are you?}
 
 
 Salut, \textbf{ça va}? (Hi! How are you?)
 
 
 \textbf{Ça va}, et toi? (I'm fine, and you?)
 
 \section{Accents}
 In French, an accent mark
 over a letter can make a new sound.

 
 je: j\textbf{e} (as in 'p\textbf{e}tition') \textit{I}.
 
 
 é: journ\textbf{é}e (similar to 'b\textbf{ay}') \textit{day}.
 
 
 è: tr\textbf{è}s (as in 'b\textbf{e}t') \textit{very}.
 
 
 If the last letter of a word is an \textbf{e} (without an accent!), it's usually silent.
 
 femm\textbf{e} \textit{woman}

 
 bonn\textbf{e} nuit! \textit(good night)
 
 \section{Verbs}
 
 
 \end{document}
